\documentclass[12pt,a4paper,oneside]{book}
\usepackage[italian]{babel}
\usepackage[UTF8]{inputenc}
%\usepackage[latin1]{inputenc}
\usepackage{amsmath}
\usepackage{amsfonts}
\usepackage{amssymb}
\usepackage{graphicx}
\usepackage{eso-pic}
\usepackage{fancyhdr} 
\newcommand{\fncyblank}{\fancyhf{}}
\newenvironment{abstract}% 
{\cleardoublepage\fncyblank\null \vfill\begin{center}% 
\bfseries \abstractname \end{center}}% 
{\vfill\null}
\newcommand\AlCentroPagina[1]{% 
\AddToShipoutPicture*{\AtPageCenter{%
\makebox(0,0){\includegraphics%
[width=0.9\paperwidth]{#1}}}}}
\setlength{\parindent}{0pt}
\setlength{\parskip}{1ex plus 0.5ex minus 0.2ex}
\author{Maria Luisa Feola}
\title{Reti Neurali}

\begin{document}
	%FRONTESPIZIO
	\AlCentroPagina{Immagini/Frontespizio}\thispagestyle{empty}
	
	%DEDICA
	\begin{flushright} 
		\newpage
		\null\vspace{\stretch{1}}
		\thispagestyle{empty}
		Dedica
		\vspace{\stretch{2}}\null
	\end{flushright} 
   	%FINE DEDICA
   	
   	%INDICE
	\tableofcontents
	%FINE INDICE
	
	%INTRODUZIONE
	\chapter*{Introduzione}
	\addcontentsline{toc}{chapter}{Introduzione}
	
	%FINE INTRODUZIONE
		
	%INIZIO DEL PRIMO CAPITOLO
	\chapter{Generale}
		\section{Cosa, quando e perchè?}
		\section{Biologia}
		\section{Struttura e modello matematico}
		\section{Architettura}
		\section{Funzioni di attivazione}
	%FINE DEL PRIMO CAPITOLO
	
	\chapter{Feed Forward}
	\chapter{Ricorrenti}

\clearpage 
%\phantomsection 
\addcontentsline{toc}{section}{\refname}
\begin{thebibliography}{hetichetta più lungai} \bibitem[hetichetta personalizzatai]{hchiave di citazionei} ... 
\end{thebibliography}
	
\end{document}